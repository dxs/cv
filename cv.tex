\documentclass[10pt,a4paper]{moderncv}
\moderncvtheme[blue]{classic}                
\usepackage[utf8]{inputenc}
\usepackage[scale=0.8]{geometry}

\firstname{Sven}
\familyname{Borden}
\title{Étudiant}              
\address{Chemin de Crausaz 18}{1162 St-Prex}    
\mobile{078 611 32 21}                    
\email{sborden@bluewin.ch}                      
%\homepage{www.ugvdrone.wordpress.com}

\begin{document}
\maketitle

\section{Diplômes et Études}
\cventry{2000 -- 2010}{Etudes obligatoire}{St-Prex}{}{}{Option Mathématiques et Physique}
\cventry{2010 -- Maintenant}{Etudes gymnasiales}{Marcelin, Morges}{}{}{Option spécifique Mathématiques et Physique, option complémentaire Chimie}
\cventry{2002 -- 2009}{Etudes de Solfège}{Conservatoire de Lausanne}{}{}{Obtenu le certificat en 2010}
\cventry{2001 -- Maintenant}{Etudes de Piano}{Conservatoire de Lausanne}{}{Nombreux spectacles et prix}{Fin de certificat en 2014}
\cventry{été 2013}{Recherches en Informatique}{Université de Lafayette, Louisianne, USA}{}{}{Implémentation de la parallélisation dans l'interpréteur Javascript}

\section{Experiences}
\cventry{Octobre 2012 et Décembre 2012}{Employé de commandes}{Bugnard SA}{Lausanne}{}{Préparation de commande, préparation d'assortiments et comptabilité\newline{}}


\section{Compétences}
\subsection{Informatique}

\cvcomputer{POO}{C++, Java, Python, Shellscript}{Web}{Javascript}
\cvcomputer{Langage impératif}{C}{Autre}{Bonne connaissance en parallélisation de programmes en C. Maitrise de \LaTeX. Bonnes connaissances des OS à base GNU/Linux.}

\subsection{Langues}
\cvlanguage{Français}{lu, parlé, écrit (C2)}{}
\cvlanguage{Anglais}{lu, parlé, écrit (B2/C1)}{}
\cvlanguage{Allemand}{lu, parlé, écrit (B2)}{}
\cvlanguage{Chinois}{parlé, notion en lu et écrit}{}


\section{Centres d'intérêt}
\cvline{Sport}{Vélo, tennis, football, beach-volley et basketball}
\cvline{Informatique}{Optimisation, transfert de donnée Client/Serveur, interfaces graphiques...}
\cvline{Robotique}{Drones, systèmes autonomes...}
\cvline{Art et culture}{Piano, lecture et histoire de la musique}


\end{document}
